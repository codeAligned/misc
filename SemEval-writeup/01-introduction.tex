\section{Introduction}
	Automated classification of natural language is thriving field in computer science due to the variety of practical applications.  One of these many applications is automated analysis of social media. Due to its widespread usage across multiple demographics, social media data provides a vast quantity of information ripe for use in both industry and government sectors.  One use of automated classification is for determining whether a body of text has a positive or negative sentiment.  Sentiment evaluation has a wide variety of uses, such as identifying a person's feelings about a product, sports team or their own government.  This could allow for better targetting and classification of individuals themselves, allowing businesses and governments to know their consumers or citizens more accurately, allowing them to do their respective jobs better. 
	
   The \AlgName\ algorithm is a machine learning algorithm designed to tag raw Twitter data ('tweets') with individual sentiment tags.  This algorithm was created for Task 9 of SemEval 2014~\cite{rosen}.  This algorithm is largely SVM-based, with some added functionality based on heuristics we found useful.  \AlgName\ provided decent performance, though had some issues distinguishing between negative and neutral tweets. 